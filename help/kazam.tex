\documentstyle[a4,makeidx,verbatim,texhelp,fancyhea,mysober,mytitle]{report}%
\parskip=10pt%
\parindent=0pt%
\title{Kazam}%
\author{(c) 2003 Johan Berntsson}%
\makeindex%
\begin{document}%
\maketitle%
\pagestyle{fancyplain}%
\bibliographystyle{plain}%
\pagenumbering{roman}%
\setheader{{\it CONTENTS}}{}{}{}{}{{\it CONTENTS}}%
\setfooter{\thepage}{}{}{}{}{\thepage}%
\tableofcontents%

\chapter{Introduction}\label{introduction}
\pagenumbering{arabic}%
\setheader{{\it CHAPTER \thechapter}}{}{}{}{}{{\it CHAPTER \thechapter}}%
\setfooter{\thepage}{}{}{}{}{\thepage}%

Kazam is a visual editor and project manager for interactive game creation with Inform.
It is developed with the wxWindows cross-platform GUI package, and there will be versions
for Linux and Macintosh available once the program is stable.

Kazam operates on Inform source files. When opening a source file it is read into
memory. Every change in editors like the room editor or the property pages is directly
inserted in the memory copy of the source file. The memory copy will be save to file
when the user selects Save. The status of all source files will be checked before Build,
and any updated sources will be written to file before compiling. 

\chapter{Getting Started}\label{gettingstarted}
\pagenumbering{arabic}%
\setheader{{\it CHAPTER \thechapter}}{}{}{}{}{{\it CHAPTER \thechapter}}%
\setfooter{\thepage}{}{}{}{}{\thepage}%
Download the latest version and use Edit/Preferences to set up your Inform compiler/interpreter. Use File/Open
to load the attached test file "game.inf". 

\chapter{Notebooks}\label{notebooks}
\pagenumbering{arabic}%
\setheader{{\it CHAPTER \thechapter}}{}{}{}{}{{\it CHAPTER \thechapter}}%
\setfooter{\thepage}{}{}{}{}{\thepage}%

All editing and data presentation takes place in notebooks, specific for each Inform data type. 

\section{Room}\label{roomnotebook}
From this notebook you can add, remove and edit rooms in your game. It consists of three views:

\subsection{Map View}\label{mapview}
From the map view all rooms can be displayed and edited. The Map menu will be inserted
in the menu bar whenever this room is active, and from it you can add and delete rooms
and connections. You can also modify the canvas size and zoom in and out.

Auto arrange will relocate all rooms to locations based on their connections, as defined
by n_to etc in the objects themselves. Some manual modifications will probably be required,
but it is a good way of providing reasonable defaults when importing files that
don't have location information into Kazam.

How to:
\begin{itemize}
\item Add a room: left double click on an empty square
\item Add a connection: left click on a direction hotspot and drag to another room's hotspot
\item Add a door: press shift and add a connection
\item Select a room: left click on the room
\item Select a connection: left click on the hotspot
\item Edit a room: double left click on the room
\item Move a room: left click and drag to an empty square
\item Delete a room: select the room and choose Delete from the menu
\item Delete a connection: select the connection and choose Delete from the menu
\end{itemize}

\subsection{Properties}\label{roomproperties}
The directions, ID, dictionary words etc can be edited here.

\subsection{Source Code}\label{roomsource}
Here you can edit the source code directly.

\addcontentsline{toc}{chapter}{Index}
\setheader{{\it INDEX}}{}{}{}{}{{\it INDEX}}%
\setfooter{\thepage}{}{}{}{}{\thepage}%
\printindex%

\end{document}
